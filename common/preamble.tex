%Generic Package Imports
\usepackage[a4paper]{geometry}
\usepackage[english]{babel}
\usepackage[utf8]{inputenc}
\usepackage{subfiles}
\usepackage{booktabs, tabularx}

%Standard Math tools
\usepackage{amssymb, mathtools, enumitem} 

%Fancy Fonts
% Regular useful fonts
\usepackage{dsfont, mathrsfs}
\usepackage[T1]{fontenc}

%Change numbering for equations and figures
\usepackage{xassoccnt}

% Number within Section
\numberwithin{equation}{chapter}
\DeclareCoupledCounters[name=figeq]{figure, equation}
\CounterWithin{equation}{chapter}

%Format chapter, section, and subsection headings better
\usepackage{titlesec}
\titleformat{\section}{\normalfont\bfseries\scshape}{\thesection}{1em}{}
\titleformat{\subsection}{\normalfont\fontsize{12}{15}\scshape}{\thesubsection}{1em}{}
\titleformat{\chapter}[display]
 {\bfseries\Large\lsstyle\vspace{5cm}\filleft}
 {\MakeUppercase{\chaptername}\enspace\thechapter}
 {2ex}
 {\titlerule[1pt]\vspace{2ex}\MakeUppercase}%
\titlespacing*{\chapter}{0pt}{-60pt}{10ex}

%Better / easier to remember list environments
\newenvironment{romlist}{\begin{enumerate}[topsep=0pt,itemsep=-1ex,partopsep=1ex,parsep=1ex,label={(\roman*)}]}{\end{enumerate}}
\newenvironment{numlist}{\begin{enumerate}[topsep=0pt,itemsep=-1ex,partopsep=1ex,parsep=1ex,label={(\arabic*)}]}{\end{enumerate}}
\newenvironment{abclist}{\begin{enumerate}[topsep=0pt,itemsep=-1ex,partopsep=1ex,parsep=1ex,label={(\alph*)}]}{\end{enumerate}}
\newenvironment{properties}{\begin{numlist}}{\end{numlist}}
\newenvironment{parts}{\begin{abclist}}{\end{abclist}}

%Bibliography setup (with BibLaTeX + Biber)
\usepackage{hyperref}
\hypersetup{
    linktoc=all,
    colorlinks=true,
    linkcolor=green!35!black, 
    anchorcolor=green!35!black, 
    citecolor=magenta!45!black,
    urlcolor=cyan!65!black}
\usepackage[nameinlink]{cleveref}
\newcommand*{\fullref}[1]{\hyperref[{#1}]{\Cref*{#1} \nameref*{#1}}}
\usepackage[style=numeric]{biblatex}
\addbibresource{references.bib}

%Graphics Packages and shortcuts
\usepackage{graphicx}
\usepackage{import}
\usepackage{xifthen}
\usepackage{pdfpages}
\usepackage{transparent}
\newcommand{\incfig}[1]{\def\svgwidth{\columnwidth}\import{./lectures/figures/}{#1.pdf_tex}}

%TikZ Packages
\usepackage{tikz}
\usepackage{tikz-cd}
\usepackage{tikzsymbols}

%Endnotes
\usepackage{enotez}
\setenotez{
    list-heading={\section{#1}}
}
\let\footnote=\endnote


%Break-plain theorem environments
\usepackage[amsmath, thmmarks, hyperref, amsthm]{ntheorem}
\usepackage{thmtools}
\usepackage[framemethod=tikz]{mdframed}

%AMS Theorem Style Environments
\newtheoremstyle{ritz-break}%
  {\item[\rlap{\vbox{\hbox{\hskip\labelsep \theorem@headerfont
          ##1\ ##2\theorem@separator}\hbox{\strut}}}]}%
  {\item[\rlap{\vbox{\hbox{\hskip\labelsep \theorem@headerfont 
          ##1\ ##2\ : ##3\theorem@separator}\hbox{\strut}}}]}

\newtheoremstyle{ritz-prob}%
  {\item[\rlap{\vbox{\hbox{\hskip\labelsep \theorem@headerfont
          ##1\theorem@separator}\hbox{\strut}}}]}%
  {\item[\rlap{\vbox{\hbox{\hskip\labelsep \theorem@headerfont 
          ##1\ ##3\theorem@separator}\hbox{\strut}}}]}

\theoremstyle{break}
\theoremindent=0.3cm
\theoremheaderfont{\kern-0.3cm\smallskip\normalfont\bfseries}

\newtheorem{theorem}[equation]{Theorem}
\newtheorem{lemma}[equation]{Lemma}
\newtheorem{corollary}[equation]{Corollary}
\newtheorem{proposition}[equation]{Proposition}
\newtheorem{conjecture}[equation]{Conjecture}

\renewtheorem*{theorem*}{Theorem}
\renewtheorem*{proposition*}{Proposition}
\renewtheorem*{corollary*}{Corollary}
\renewtheorem*{lemma*}{Lemma}

%AMS Definition Style Environments
\theorembodyfont{\normalfont\upshape}
\newtheorem{definition}[equation]{Definition}
\newtheorem{example}[equation]{Example}
\newtheorem{algorithm}[equation]{Algorithm}
\newtheorem{axiom}[equation]{Axiom}
\newtheorem{property}[equation]{Property}

\renewtheorem*{definition*}{Definition}
\renewtheorem*{example*}{Example}
\renewtheorem*{algorithm*}{Algorithm}
\renewtheorem*{property*}{Property}


% Homework problem counters 
\theoremstyle{break}
\newtheorem*{exercise}{Exercise}
\newtheorem*{problem}{Problem}


%AMS Remark/Notation Style Environments
\theoremheaderfont{\kern-0.2cm\smallskip\normalfont\scshape}
\newtheorem{remark}[equation]{Remark}
\newtheorem{note}[equation]{Note}
\newtheorem{notation}[equation]{Notation}


%Bolder proof & solutions environment
\makeatletter
\renewenvironment{proof}[1][\proofname]
{\par
\normalfont\topsep6\p@\@plus6\p@\relax\trivlist
\item[\hskip\labelsep\bfseries\itshape#1\@addpunct{.}]
\mbox{}
\par
\nobreak}
{
\begin{flushright}
\ensuremath{\blacksquare}
\end{flushright}
\endtrivlist\@endpefalse}
\makeatother
\newenvironment{solution}{\begin{proof}[Solution]}{\end{proof}}
\newenvironment{outline}{\begin{proof}[Outline of the Proof]\begin{numlist}}{\end{numlist}\end{proof}}

%Swap phi -> varphi and epsilon -> varepsilon
\newcommand{\oldepsilon}{\epsilon}
\renewcommand{\epsilon}{\varepsilon}
\newcommand{\oldphi}{\phi}
\renewcommand{\phi}{\varphi}

% Default macros for working with math
% Standard commands for Mathbb letters
\newcommand{\NN}{\mathds{N}}
\newcommand{\ZZ}{\mathds{Z}}
\newcommand{\ZZps}{\mathds{Z}_{>0}}	 % ZZps = Z (strictly) positive
\newcommand{\ZZnn}{\mathds{Z}_{\ge 0}}	 % ZZnn = Z non-negative
\newcommand{\QQ}{\mathds{Q}}
\newcommand{\RR}{\mathds{R}}
\newcommand{\RRps}{\mathds{R}_{>0}}
\newcommand{\RRnn}{\mathds{R}_{\ge 0}}
\newcommand{\CC}{\mathds{C}}
\newcommand{\Sn}{\mathds{S}^{n-1}}
\newcommand{\Sphere}[1]{\mathds{S}^{#1}}


% Standard commands for Mathbb letters
\newcommand{\Abb}{\mathbb{A}}
\newcommand{\Bbb}{\mathbb{B}}
\newcommand{\Dbb}{\mathbb{D}}
\newcommand{\Ebb}{\mathbb{E}}
\newcommand{\Fbb}{\mathbb{F}}
\newcommand{\Gbb}{\mathbb{G}}
\newcommand{\Hbb}{\mathbb{H}}
\newcommand{\Kbb}{\mathbb{K}}
\newcommand{\Lbb}{\mathbb{L}}
\newcommand{\Mbb}{\mathbb{M}}
\newcommand{\Nbb}{\mathbb{N}}
\newcommand{\Rbb}{\mathbb{R}}
\newcommand{\Sbb}{\mathbb{S}}
\newcommand{\Tbb}{\mathbb{T}}
\newcommand{\Ubb}{\mathbb{U}}
\newcommand{\Vbb}{\mathbb{V}}
\newcommand{\Wbb}{\mathbb{W}}
\newcommand{\Xbb}{\mathbb{X}}
\newcommand{\Ybb}{\mathbb{Y}}
\newcommand{\Zbb}{\mathbb{Z}}

% Standard commands for Mathds letters
\newcommand{\Ads}{\mathds{A}}
\newcommand{\Bds}{\mathds{B}}
\newcommand{\Dds}{\mathds{D}}
\newcommand{\Eds}{\mathds{E}}
\newcommand{\Fds}{\mathds{F}}
\newcommand{\Gds}{\mathds{G}}
\newcommand{\Hds}{\mathds{H}}
\newcommand{\Kds}{\mathds{K}}
\newcommand{\Lds}{\mathds{L}}
\newcommand{\Mds}{\mathds{M}}
\newcommand{\Nds}{\mathds{N}}
\newcommand{\Rds}{\mathds{R}}
\newcommand{\Sds}{\mathds{S}}
\newcommand{\Tds}{\mathds{T}}
\newcommand{\Uds}{\mathds{U}}
\newcommand{\Vds}{\mathds{V}}
\newcommand{\Wds}{\mathds{W}}
\newcommand{\Xds}{\mathds{X}}
\newcommand{\Yds}{\mathds{Y}}
\newcommand{\Zds}{\mathds{Z}}

% Standard commands for Mathcal letters
\newcommand{\Acal}{\mathcal{A}}
\newcommand{\Bcal}{\mathcal{B}}
\newcommand{\Dcal}{\mathcal{D}}
\newcommand{\Ecal}{\mathcal{E}}
\newcommand{\Fcal}{\mathcal{F}}
\newcommand{\Gcal}{\mathcal{G}}
\newcommand{\Hcal}{\mathcal{H}}
\newcommand{\Kcal}{\mathcal{K}}
\newcommand{\Lcal}{\mathcal{L}}
\newcommand{\Mcal}{\mathcal{M}}
\newcommand{\Ncal}{\mathcal{N}}
\newcommand{\Rcal}{\mathcal{R}}
\newcommand{\Scal}{\mathcal{S}}
\newcommand{\Tcal}{\mathcal{T}}
\newcommand{\Ucal}{\mathcal{U}}
\newcommand{\Vcal}{\mathcal{V}}
\newcommand{\Wcal}{\mathcal{W}}
\newcommand{\Xcal}{\mathcal{X}}
\newcommand{\Ycal}{\mathcal{Y}}
\newcommand{\Zcal}{\mathcal{Z}}

% Standard commands for Mathscr letters 
\newcommand{\Ascr}{\mathscr{A}}
\newcommand{\Bscr}{\mathscr{B}}
\newcommand{\Cscr}{\mathscr{C}}
\newcommand{\Dscr}{\mathscr{D}}

% Probability Theory Macros
\newcommand{\prob}{\Pr}
\renewcommand{\Pr}{\mathds{P}}
\newcommand{\Ex}{\mathds{E}}
\newcommand{\given}{\mid}
\newcommand{\Ind}[1]{\mathds{1}_{#1}}
\newcommand{\IndCond}[1]{\mathds{1}_{\left\{#1\right\}}}
\DeclareMathOperator{\Var}{Var}
\DeclareMathOperator{\var}{Var}
\DeclareMathOperator{\corr}{Corr}
\DeclareMathOperator{\Corr}{Corr}
\DeclareMathOperator{\Cov}{Cov}
\DeclareMathOperator{\cov}{Cov}

% Specific Distributions
\DeclareMathOperator{\Gauss}{Gauss}
\DeclareMathOperator{\Pois}{Pois}
\DeclareMathOperator{\Unif}{Unif}
\DeclareMathOperator{\Haar}{Haar}

% Geometry Macros
\DeclareMathOperator{\Aff}{Aff}
\DeclareMathOperator{\aff}{Aff}
\DeclareMathOperator{\lin}{Lin}
\DeclareMathOperator{\Lin}{Lin}
\DeclareMathOperator{\Span}{Span}
\DeclareMathOperator{\span}{Span}
\DeclareMathOperator{\Rank}{Rank}
\DeclareMathOperator{\rank}{Rank}
\DeclareMathOperator{\nullity}{Null}
\DeclareMathOperator{\Nullity}{Null}
\DeclareMathOperator{\Range}{Range}
\DeclareMathOperator{\range}{Range}
\DeclareMathOperator{\Area}{Area}
\DeclareMathOperator{\area}{Area}
\DeclareMathOperator{\Conv}{Conv}
\DeclareMathOperator{\conv}{Conv}
\DeclareMathOperator{\Diam}{Diam}
\DeclareMathOperator{\diam}{Diam}
\DeclareMathOperator{\Vol}{Vol}
\DeclareMathOperator{\vol}{Vol}
\DeclareMathOperator{\Proj}{Proj}
\DeclareMathOperator{\proj}{Proj}
\DeclareMathOperator{\perim}{Perim}
\DeclareMathOperator{\Perim}{Perim}
\DeclareMathOperator{\Frob}{Frob}
\DeclareMathOperator{\HS}{HS}
\newcommand{\polar}{^{\circ}}
\DeclareMathOperator{\relintr}{rel. int.}
\DeclareMathOperator{\relbd}{rel. bd.}
\DeclareMathOperator{\relcl}{rel. cl.}
\DeclareMathOperator{\bd}{bd.}
\DeclareMathOperator{\intr}{int.}
\DeclareMathOperator{\cl}{cl.}

% Measure Theory Macros
\DeclareMathOperator{\esssup}{ess\,sup}
\DeclareMathOperator{\essinf}{ess\,inf}
\DeclareMathOperator{\supp}{supp}
\DeclareMathOperator{\esssupp}{ess\,supp}
\newcommand{\io}{\text{ i.o.}}
\newcommand{\ult}{\text{ ult.}}
\newcommand{\ae}{\text{ a.e.}}
\newcommand{\as}{\text{ a.s.}}

% Linear and Functional Analysis Macros
\DeclareMathOperator{\Id}{Id}
\DeclareMathOperator{\TV}{TV}
\DeclareMathOperator{\eigvals}{Eig.\,Vals.}
\DeclareMathOperator{\eigvecs}{Eig.\,Vecs.}
\DeclareMathOperator{\eigfuncs}{Eig.\,Funcs.}
\newcommand{\sa}{^{\mathsf{sa}}}
\newcommand{\transpose}{^{\intercal}}
\newcommand{\htranspose}{^{\mathsf{H}}}
\newcommand{\adjoint}{^{\mathsf{*}}}
\newcommand{\dual}{^{\mathsf{*}}}
\newcommand{\sstar}{^{\mathsf{*}}}
\newcommand{\inverse}{^{-1}}
\newcommand{\psinverse}{^{\dagger}}

% Integral Macros
\newcommand*{\dd}{\mathop{}\!\mathrm{d}}
\newcommand*{\ds}{\dd s}
\newcommand*{\dt}{\dd t}
\newcommand*{\du}{\dd u}
\newcommand*{\dv}{\dd v}
\newcommand*{\dx}{\dd x}
\newcommand*{\dy}{\dd y}
\newcommand*{\dz}{\dd z}
\newcommand*{\dr}{\dd r}
\newcommand*{\dw}{\dd \omega}
\newcommand*{\dth}{\dd \theta}
\newcommand*{\mudw}{\mathop{}\!\mu(\mathrm{d}\omega)}
\newcommand*{\mudx}{\mathop{}\!\mu(\mathrm{d}x)}
\newcommand*{\mudy}{\mathop{}\!\mu(\mathrm{d}y)}
\newcommand*{\mudz}{\mathop{}\!\mu(\mathrm{d}z)}
\newcommand*{\nudw}{\mathop{}\!\nu(\mathrm{d}\omega)}
\newcommand*{\nudx}{\mathop{}\!\nu(\mathrm{d}x)}
\newcommand*{\nudy}{\mathop{}\!\nu(\mathrm{d}y)}
\newcommand*{\nudz}{\mathop{}\!\nu(\mathrm{d}z)}
\newcommand*{\lmdw}{\mathop{}\!\lambda(\mathrm{d}\omega)}
\newcommand*{\lmdx}{\mathop{}\!\lambda(\mathrm{d}x)}
\newcommand*{\lmdy}{\mathop{}\!\lambda(\mathrm{d}y)}
\newcommand*{\lmdz}{\mathop{}\!\lambda(\mathrm{d}z)}
\newcommand*{\gmdw}{\mathop{}\!\gamma(\mathrm{d}\omega)}
\newcommand*{\gmdx}{\mathop{}\!\gamma(\mathrm{d}x)}
\newcommand*{\gmdy}{\mathop{}\!\gamma(\mathrm{d}y)}
\newcommand*{\gmdz}{\mathop{}\!\gamma(\mathrm{d}z)}
\newcommand*{\prdw}{\mathop{}\!\Pr(\mathrm{d}\omega)}
\newcommand*{\prdx}{\mathop{}\!\Pr(\mathrm{d}x)}
\newcommand*{\prdy}{\mathop{}\!\Pr(\mathrm{d}y)}
\newcommand*{\prdz}{\mathop{}\!\Pr(\mathrm{d}z)}
\newcommand*{\sgdth}{\mathop{}\!\sigma(\mathrm{d}\theta)}
\newcommand*{\Hdx}{\mathop{}\!\mathcal{H}(\mathrm{d}x)}
\newcommand*{\Hdy}{\mathop{}\!\mathcal{H}(\mathrm{d}y)}
\newcommand*{\Hdz}{\mathop{}\!\mathcal{H}(\mathrm{d}z)}

% Derivative Macros
\newcommand*{\pd}{\partial}
\newcommand*{\grad}{\nabla}
\newcommand{\divisionsymbol}{\div}
\renewcommand*{\div}{\nabla\cdot}
\newcommand*{\curl}{\nabla\times}
\newcommand*{\laplacian}{\nabla^2}
\newcommand*{\DirD}{\mathbf{D}}
\newcommand*{\hessian}{\mathbf{H}}
\newcommand*{\jacobian}{\mathbf{J}}

% Algebra & Category Theory Macros
\newcommand{\op}{^{\mathsf{op}}}
\newcommand{\Cat}[1]{\mathsf{#1}}
\DeclareMathOperator{\coker}{coker}
\DeclareMathOperator{\codim}{codim}
\DeclareMathOperator{\Hom}{Hom}
\DeclareMathOperator{\End}{End}
\DeclareMathOperator{\Sym}{Sym}
\DeclareMathOperator{\Aut}{Aut}
\DeclareMathOperator{\Gal}{Gal}
\DeclareMathOperator{\Stab}{Stab}
\DeclareMathOperator{\Orb}{Orb}
\DeclareMathOperator{\OrthogonalGr}{O}
\DeclareMathOperator{\UnitaryGr}{U}
\DeclareMathOperator{\GL}{GL}
\DeclareMathOperator{\SL}{SL}
\DeclareMathOperator{\Sp}{Sp}
\DeclareMathOperator{\SO}{SO}
\DeclareMathOperator{\SU}{SU}
\DeclareMathOperator{\PGL}{PGL}
\DeclareMathOperator{\PSL}{PSL}
\DeclareMathOperator{\PSO}{PSO}
\DeclareMathOperator{\PSU}{PSU}

% Set Theory Macros
\newcommand{\comp}{^{\mathsf{c}}}
\newcommand{\Set}[2]{\left\{#1\;\middle\mid\;#2\right\}}
\DeclareMathOperator{\card}{Card}
\DeclareMathOperator{\Card}{Card}
\newcommand{\pset}[1]{2^{#1}}

% Macros for Limit Shorthands
\newcommand{\toUnif}{\xrightarrow{\text{unif}}}
\newcommand{\toAs}{\xrightarrow{\text{a.s.}}}
\newcommand{\toAe}{\xrightarrow{\text{a.e.}}}
\newcommand{\toMeas}{\xrightarrow{\text{in meas.}}}
\newcommand{\toWeak}{\xrightarrow{\text{weak}}}
\newcommand{\toWeakSt}{\xrightarrow{\text{weak}-*}}
\newcommand{\toDist}{\xrightarrow{\text{in dist.}}}
\newcommand{\upto}{\nearrow}
\newcommand{\downto}{\searrow}

%Math operators otherwise not defined
\DeclareMathOperator{\argmax}{arg\,max}
\DeclareMathOperator{\argmin}{arg\,min}
\DeclareMathOperator{\lcm}{lcm}
\DeclareMathOperator{\real}{Re}
\DeclareMathOperator{\imag}{Im}
\DeclareMathOperator{\sign}{sign}
\DeclareMathOperator{\Res}{Res}
\DeclareMathOperator{\tr}{tr}
\DeclareMathOperator{\Tr}{tr}

% Generic Text Manipulation Macros
\newcommand{\txand}{\text{ and }}
\newcommand{\txor}{\text{ or }}
\newcommand{\txforany}{\text{ for any }}
\newcommand{\txforsome}{\text{ for some }}
\newcommand{\caseif}{& \text{ if }}
\newcommand{\caseotherwise}{& \text{ otherwise}}
\newcommand{\SPC}{\mathop{}\!}

%Replace non-wide with wide when available
\newcommand{\narrowbar}{\bar}
\newcommand{\narrowhat}{\hat}
\newcommand{\narrowtilde}{\tilde}
\renewcommand{\bar}{\overline}
\renewcommand{\hat}{\widehat}
\renewcommand{\tilde}{\widetilde}

% Fix minor font issues
\let\oldvec=\vec
\renewcommand{\vec}[1]{\oldvec{\mathbf{#1}}}
\newcommand{\ambigsubset}{\subset}
\newcommand{\ambigsupset}{\supset}
\renewcommand{\subset}{\subsetneq}
\renewcommand{\supset}{\supsetneq}

% Some standard environments
\newcommand{\norm}[1]{\left\lVert #1 \right\rVert}
\newcommand{\abs}[1]{\left\lvert #1 \right\rvert}
\newcommand{\floor}[1]{\left\lfloor #1 \right\rfloor}
\newcommand{\ceil}[1]{\left\lceil #1 \right\rceil}
\newcommand{\bra}[1]{\left\langle #1 \right\rvert}
\newcommand{\ket}[1]{\left\lvert #1 \right\rangle}
\newcommand{\braket}[2]{\left\langle #1 \middle\mid #2\right\rangle}
\newcommand{\ketbra}[2]{\left\lvert #1\middle\rangle\middle\langle #2\right\rvert}
\newcommand{\braketA}[3]{\left\langle #1 \middle\mid #2 \middle\mid #3\right\rangle}
\newcommand{\innerp}[1]{\left\langle #1 \rangle}
\newcommand{\Generates}[2]{\left\langle #1 \;\middle\mid\; #2 \right\rangle}
\newcommand{\BigO}[1]{\mathcal{O}\left(#1\right)}
\newcommand{\LittleO}[1]{\mathcal{o}\left(#1\right)}
\newcommand{\BigTheta}[1]{\Theta\left(#1\right)}
\newcommand{\BigOmega}[1]{\Omega\left(#1\right)}

% Quantum Mechanics Specific Characters
\newcommand{\upbra}{\bra{\uparrow}}
\newcommand{\dnbra}{\bra{\downarrow}}
\newcommand{\upket}{\ket{\uparrow}}
\newcommand{\dnket}{\ket{\downarrow}}
\newcommand{\phibra}{\bra{\phi}}
\newcommand{\phiket}{\ket{\phi}}
\newcommand{\psibra}{\bra{\psi}}
\newcommand{\psiket}{\ket{\psi}}
